\section{Задание 2. Кривые второго порядка.}

\textbf{Условие.}

Даны уравнения двух множеств:

Множество 1: $40x^2 + 36xy + 25y^2 - 8x - 14y + 1 = 0$

Множество 2: $5x^2 + 4xy - y^2 - 5x + y = 0$

\begin{enumerate}
    \item Покажите, что одно из множеств является кривой второго порядка, сведя его
уравнение к каноническому виду преобразованием координат, а другое – кривой,
распавшейся на прямые (найдите уравнения прямых).
    \item Изобразите каждое множество на отдельном рисунке вместе со старой и новой
системой координат (оси новой системы должны служить осями симметрии
множества).
    \item У нераспавшейся кривой определите расстояние $p$ между фокусом и директрисой и
эксцентриситет $\varepsilon$. Запишите полярное уравнение кривой с найденными параметрами.
    \item На одном рисунке совместите началами и осями $Ox$ декартову прямоугольную и
полярную системы координат. Постройте кривую по ее каноническому уравнению в
ДПСК и по ее полярному уравнению в ПСК. Объясните несовпадение кривых.
    \item Найдите такое расположение ПСК и формулы преобразования полярных координат в
декартовы, чтобы полярное и каноническое уравнения описывали одну и ту же
кривую.
\end{enumerate}
\vspace{10mm}
\textbf{Решение.}

It is empty but you can fill it!

\textit{Ответ}:  It is empty but you can fill it!
\clearpage